\documentclass{beamer}

\mode<presentation>

\title{Write You A Garbage Collector: The Basics}
\author{Matthew Russo}
\institute{Reinventing the Wheel}
\date{February 12th, 2020}

\begin{document}
 
\frame{\titlepage}

\begin{frame}
  \frametitle{Who Am I?}
  \begin{enumerate}
    \item Consistently reinventing the wheel
    \item Enjoy listening to and discovering new music. Would like to one day make music!
    \item Software Engineer for AWS working on ECR
    \item Knowledge sharing around these more `mysterious' parts of software development.
  \end{enumerate}
\end{frame}

\begin{frame}
  \frametitle{Who Are You?}
  \begin{enumerate}
    \item Name & Pronouns
    \item Why you joined this group?
    \item Any specific topics of interest? Projects you've worked on?
    \item Anything else you'd like us to know
  \end{enumerate}
\end{frame}

\begin{frame}
  \frametitle{Why are we here? (and what are we not here for)}
\end{frame}

\begin{frame}
  \frametitle{Terminology}
  \begin{enumerate}
    \item Stack & Heap \- Two places to store things
    \item Runtime \- Invisible interfaces that handle everything between the Operating System and the User Code
    \item Allocator \- Finds places to store things
    \item Collector \- Reclaims places to store things
    \item Mutator \- User code \-\- interacts with Runtime to make things happen
    \item Root \- Entry in to the object graph. Stack variables, frame locals, global variables, etc.
  \end{enumerate}
\end{frame}

\begin{frame}
  \frametitle{Four basic garbage collection algorithm families}
  \begin{enumerate}
    \item Mark-Sweep (Tracing)
    \item Mark-Compact (Tracing)
    \item Copying (Tracing)
    \item Reference Counting
  \end{enumerate}
\end{frame}

\begin{frame}
  \frametitle{Marking in Tracing collectors}
  \begin{enumerate}
    \item Depth-first Search through the object graph
    \item Tricolor abstraction
    \item The Object Header
  \end{enumerate}
\end{frame}

\begin{frame}
  \frametitle{Mark-Sweep Collection}
  \begin{itemize}
    \item The Heap as Iterator
    \begin{enumerate}
      \item Step 1. Mark the Object Graph
      \item Step 2. Iterate over the heap
      \item Step 3. If the Object hasn't been marked: Delete it
    \end{enumerate}
  \end{itemize}
\end{frame}

\begin{frame}
  \frametitle{Mark-Compact Collection}
  \item Handling Fragmented Memory
  \begin{enumerate}
    \item Step 1. Mark the Object Graph
    \item Step 2. Iterate over the heap
    \item Step 3. If the Object hasn't been marked: Delete it
    \item Step 3. If the Object has    been marked: Move it to the first free space in the heap
  \end{enumerate}
\end{frame}

\begin{frame}
  \frametitle{Copying Collection}
  \item Handling Fragmented Memory but I'm a little bit lazier
  \begin{enumerate}
    \item Step 1. During Intialization: Split the heap in two
    \item Step 2. Mark the Object Graph
    \item Step 3. Iterate over the heap
    \item Step 4. If the Object hasn't been marked: Do nothing
    \item Step 5. If the Object has    been marked: Move it to the first free space in the opposite heap space
    \item Step 6. Flip the two heap spaces
  \end{enumerate}
\end{frame}

\begin{frame}
  \frametitle{Reference Counting Collection}
  \item But why not ++ and \-\-
\end{frame}

\begin{frame}
  \frametitle{A Unified Theory of Garbage Collection}
\end{frame}

\end{document}
