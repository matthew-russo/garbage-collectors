\documentclass{beamer}

\mode<presentation>

\title{Write You A Garbage Collector: The Basics}
\author{Matthew Russo}
\institute{Reinventing the Wheel}
\date{February 12th, 2020}

\begin{document}
 
\frame{\titlepage}

\begin{frame}
  \frametitle{Who Am I?}
  \begin{enumerate}
    \item Consistently reinventing the wheel
    \item Enjoy listening to and discovering new music. Would like to one day make music!
    \item Software Engineer for AWS working on ECR
    \item Knowledge sharing around these more `mysterious' parts of software development.
  \end{enumerate}
\end{frame}

\begin{frame}
  \frametitle{Who Are You?}
  \begin{enumerate}
    \item Name \& Pronouns
    \item Why you joined this group?
    \item Any specific topics of interest? Projects you've worked on?
    \item Anything else you'd like us to know
  \end{enumerate}
\end{frame}

\begin{frame}
  \frametitle{Terminology}
  \begin{enumerate}
    \item Stack \& Heap \textendash{} Two places to store things
    \item Runtime \textendash{} Invisible interfaces that handle everything between the Operating System and the User Code
    \item Allocator \textendash{} Finds places to store things
    \item Collector \textendash{} Reclaims places to store things
    \item Mutator \textendash{} User code that interacts with Runtime to make things happen
    \item Root \textendash{} Entry in to the object graph. Stack variables, frame locals, global variables, etc.
  \end{enumerate}
\end{frame}

\begin{frame}
  \frametitle{Four basic garbage collection algorithm families}
  \begin{enumerate}
    \item Mark-Sweep (Tracing)
    \item Mark-Compact (Tracing)
    \item Copying (Tracing)
    \item Reference Counting
  \end{enumerate}
\end{frame}

\begin{frame}
  \frametitle{Marking in Tracing collectors}
  \begin{enumerate}
    \item Depth-first Search through the object graph
    \item Tricolor abstraction
    \item The Object Header
  \end{enumerate}
\end{frame}

\begin{frame}
  \frametitle{Mark-Sweep Collection}
  \begin{itemize}
    \item The Heap as Iterator
    \begin{enumerate}
      \item Step 1. Mark the Object Graph
      \item Step 2. Iterate over the heap
      \item Step 3. If the Object hasn't been marked: Delete it
    \end{enumerate}
  \end{itemize}
\end{frame}

\begin{frame}
  \frametitle{Concerns?}
  \begin{itemize}
    \item What will go wrong with this over time?
    \pause{}
    \item Fragmentation!
    %\begin{figure}
    %\begin{tikzpicture}[scale=1.8, auto,swap]
    %    % Draw a 7,11 network
    %    % First we draw the vertices
    %    \foreach \pos/\name in {{(0,2)/a}, {(2,1)/b}, {(4,1)/c},
    %                            {(0,0)/d}, {(3,0)/e}, {(2,-1)/f}, {(4,-1)/g}}
    %        \node[vertex] (\name) at \pos {$\name$};
    %    % Connect vertices with edges and draw weights
    %    \foreach \source/ \dest /\weight in {b/a/7, c/b/8,d/a/5,d/b/9,
    %                                         e/b/7, e/c/5,e/d/15,
    %                                         f/d/6,f/e/8,
    %                                         g/e/9,g/f/11}
    %        \path[edge] (\source) -- node[weight] {$\weight$} (\dest);
    %    % Start animating the vertex and edge selection. 
    %    \foreach \vertex / \fr in {d/1,a/2,f/3,b/4,e/5,c/6,g/7}
    %        \path<\fr-> node[selected vertex] at (\vertex) {$\vertex$};
    %    % For convenience we use a background layer to highlight edges
    %    % This way we don't have to worry about the highlighting covering
    %    % weight labels. 
    %    \begin{pgfonlayer}{background}
    %        \pause
    %        \foreach \source / \dest in {d/a,d/f,a/b,b/e,e/c,e/g}
    %            \path<+->[selected edge] (\source.center) -- (\dest.center);
    %        \foreach \source / \dest / \fr in {d/b/4,d/e/5,e/f/5,b/c/6,f/g/7}
    %            \path<\fr->[ignored edge] (\source.center) -- (\dest.center);
    %    \end{pgfonlayer}
    %\end{tikzpicture}
    %\end{figure}
  \end{itemize}
\end{frame}

\begin{frame}
  \frametitle{Mark-Compact Collection}
  \begin{heading}
    Handling Fragmented Memory
  \end{heading}

  \begin{enumerate}
    \item Step 1. Mark the Object Graph
    \item Step 2. Iterate over the heap
    \item Step 3. If the Object hasn't been marked: Delete it
    \item Step 3. If the Object has    been marked: Move it to the first free space in the heap
  \end{enumerate}
\end{frame}

\begin{frame}
  \frametitle{Copying Collection}
  \begin{heading}
    Handling Fragmented Memory but I'm a little bit lazier
  \end{heading}

  \begin{enumerate}
    \item Step 1. During Intialization: Split the heap in two
    \item Step 2. Mark the Object Graph
    \item Step 3. Iterate over the heap
    \item Step 4. If the Object hasn't been marked: Do nothing
    \item Step 5. If the Object has    been marked: Move it to the first free space in the opposite heap space
    \item Step 6. Flip the two heap spaces
  \end{enumerate}
\end{frame}

\begin{frame}
  \frametitle{Reference Counting Collection}

  \begin{heading}
    But why not ++ and \textendash{} \textendash{}
  \end{heading}

  \item Everytime an object is referenced, increase its reference count, when it stops being referenced, decrease that count
  \item Read/Write Barriers
  \item Write Barries now increment and decrement reference counts of objects
\end{frame}

\begin{frame}
  \frametitle{Problems?}
  \begin{enumerate}
      \pause{}
    \item Cyclic references
      \pause{}
    \item Unpredictable, recursive collections
  \end{enumerate}
\end{frame}

\begin{frame}
  \frametitle{A Unified Theory of Garbage Collection}
  
  \begin{enumerate}
    \item TODO
  \end{enumerate}
\end{frame}


\begin{frame}
  \frametitle{What Next?}
  
  \begin{enumerate}
    \item Concurrent collection
    \item Generational collection
    \item Production GC systems: JVM (CMS, G1GC, ZGC, etc.), Go, CLR (???)
    \item Monitoring/Tuning GC systems: what metrics are available, how to analyze performance, what parameters to tune
  \end{enumerate}
\end{frame}

\begin{frame}
  \frametitle{Questions for me?}
\end{frame}

\begin{frame}
  \frametitle{Questions for you}

  \begin{enumerate}
    \item Medium
    \item Time/Space
    \item Topics
    \item Speakers
  \end{enumerate}
\end{frame}

\end{document}
